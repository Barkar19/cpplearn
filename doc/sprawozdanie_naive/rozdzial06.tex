\chapter{Wnioski}
\section{Dyskretyzacja}
W przypadku zbioru seeds, dyskretyzacja obiema metodami nie przyniosła znacznych różnic w skuteczności klasyfikatorów. Obie metody wykazały się podobną skutecznością. Największą dokładność (accuracy) uzyskano dla dyskretyzacji wszystkich atrybutów pomiędzy 14 wartości. Wyniosła ona odpowiednio 0.937 dla metody częstotliwościowej oraz 0.927 dla metody interwałowej. W przypadku zbyt niskiej ilości przedziałów dyskretyzacji (np. 2,3) odnotowano znaczący spadek skuteczności klasyfikatorów, co jest zgodne z intuicją - klasyfikacja będzie skuteczniejsza gdy wartości atrybutów będą niosły ze sobą większą ilość informacji.

W zbiorze ecoli dla precyzji (precision) oraz fscore wielokrotnie wystąpiła wartość 'nan' oznaczająca dosłownie nie-liczbę. W przypadku precyzji wartość ta oznacza, że wszystkie wystąpienia danej klasy zostały zaklasyfikowane jako negatywne, co oznacza, że nie można w tym przypadku określić zachowania klasyfikatora dla pozytywnych wartości predykcji. W związku z tym, że fscore oblicza się na podstawie precision oraz recall, wartość ta nie mogła zostać obliczona. Największą dokładność (accuracy) uzyskano dla dyskretyzacji wszystkich atrybutów pomiędzy 14 wartości. Wyniosła ona 0.945 dla obu metod dyskretyzacji.

Dla ostatniego zbioru - ukm, również w kilku miejscach pojawiła się wartość 'nan'. Powód jej wystąpienia jest opisany powyżej. Największą wartość dokładności uzyskano dla 8 wartości dyskretnych. Wyniosła ona około 0.915 dla obu metod.

\section{Kroswalidacja}
Do porównania wyników kroswalidacji użyto najlepszych wyników klasyfikatorów dla danej metody dyskretyzacji i ilość przedziałów. Dla wszystkich wyników obie metody dały podobne rezultaty. Jednym z powodów może być dosyć równomierna dystrybucja klas w badanych zbiorach (w szczególności zbiór seeds). Warto zauważyć, że w przypadku zwykłego rodzaju kroswalidacji możliwe jest wystąpienie wartości 'nan' dla recall, co dla kroswalidacji stratyfikowanej jest niemożliwe. Dzieje się tak dlatego, gdyż kroswalidacja stratyfikowana zapewnia równomierny rozkład atrybutów dla zbiorów testowych oraz przynajmniej 1 wystąpienie każdej z klas.
\newpage
\section{Ocena klasyfikatorów}
W przypadku trzech metod uśredniania statystyk, uśrednianie arytmetyczne i ważone zachowywało się podobnie dla zbiorów danych z dużą ilością elementów. W przypadku zbioru ecoli, gdzie kilka klas było małolicznych, widać wyraźne różnice we wskaźniku recall. Dla średniej arytmetycznej recall wyniósł 0.524, a dla średniej ważonej 0.809. Dzięki temu wyraźnie widać, jak duży wpływ na ten wskaźnik mają klasy o niewielkiej ilości elementów, czego powiedzieć nie można o wskaźniku accuracy, gdzie oba wyniki wyniosły około 0.95. Globalna metoda liczenia statystyk, zachowywała się odrobinę inaczej od pozostałych. Dla wskaźników precision oraz recall, fscore dawała wyższe wartości, a dla accuracy niższe.

\section{Rozkład normalny}
Klasyfikatory wykorzystujące aproksymację prawdopodobieństw z rozkładu normalnego charakteryzowały się niższą dokładnością dla zbiorów seeds oraz ecoli (o odpowiednio 2\% oraz 28\%). Różnice w wynikach mogą świadczyć o tym, że niektóre z atrybutów nie posiadały rozkładu normalnego. Dla zbioru ukm, wartości te były nieznacznie wyższe, zatem można przypuszczać, że atrybuty w zbiorze testowym zachowywały rozkład normalny.