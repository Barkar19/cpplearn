\chapter{Działanie klasyfikatora}
	\section{Ładowanie i przetwarzanie danych}\label{discret}
Dane do nauki klasyfikatora zapisane są w formacie tekstowym, z których każda wartość odseparowana jest od siebie przecinkiem. Po załadowaniu wszystkich elementów, rozmieszczane są one losowo w tablicy. Po załadowaniu danych programista może określić jaką metodę dyskretyzacji chciałby użyć. Poniżej przedstawiono typy zaimplementowanych dyskretyzacji.
\begin{table}[H]
\centering
\caption{Rodzaje dyskretyzacji.}
\label{dist-ukm}
\begin{tabular}{  ll m{8cm}}
NAZWA & KOD & OPIS\\\hline
Interwałowa & interval & Podział danych równomiernie pomiędzy odpowiednie zakresy liczbowe.\\\hline
Częstotliwościowa & frequency & Podział danych równomiernie pomiędzy odpowiednie zakresy częstotliwościowe. \\\hline
\end{tabular}
\end{table}
	\section{Budowanie klasyfikatora}
Po załadowaniu danych następuje budowanie klasyfikatora. Do nauki zbioru danych wykorzystuje się metodę, która implementuje algorytm opisany w \ref{algo}. 

Jeżeli zostanie ustawiona odpowiednia opcja, poszczególne atrybuty rzeczywiste zostaną potraktowane jako wartości ciągłe o odpowiednim rozkładzie normalnym. W przeciwnym razie wartości ciągłe zostaną poddane dyskretyzacji zgodnie z wybraną metodą, opisaną w \ref{discret}
	\section{Kroswalidacja}
Zaimplementowane zostały dwa sposoby wykonywania kroswalidacji:
\begin{itemize}
\item $K$-krotna walidacja krzyżowa
\item Kroswalidacja stratyfikowana
\end{itemize}

Pierwsza z nich dzieli zbiór wejściowy na $K$ części, w kolejnych iteracjach każda z nich jest zbiorem testowym, a pozostałe części zbiorem uczącym.

Druga z nich to połączenie powyższego z warunkiem zapewniającym, że w każdej części znajdzie się proporcjonalna ilość rekordów danej klasy co w zbiorze wejściowym.
	\section{Obliczenie statystyk}
W celu porównania klasyfikatorów między sobą oraz oceny ogólnej charakterystyki dla danego zbioru danych, zaimplementowano cztery statystyki:
\begin{itemize}
\item precision
\item recall
\item accuracy
\item fscore
\end{itemize}

Ich uśrednianie wynikające z zastosowania kroswalidacji zostało zaimplementowane na 3 sposoby przedstawione w tabeli \ref{avg-stats}

\begin{table}[H]
\centering
\caption{Rodzaje uśredniania.}
\label{avg-stats}
\begin{tabular}{  ll m{10cm}}
NAZWA & KOD & OPIS\\\hline
Arytmetyczna & u & Średnia arytmetyczna wskaźników dla każdej klasy.\\\hline
Ważona & w & Średnia ważona wskaźników dla każdej klasy. Wagi proporcjonalne do wystąpień klas w zbiorze. \\\hline
Globalna & g & Globalne obliczanie statystyk z macierzy konfuzji.\\\hline
\end{tabular}
\end{table}