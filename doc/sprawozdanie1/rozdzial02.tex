\chapter{Dane testowe - charakterystyka}

\section{Seeds}
	\subsection{Opis}
Zbiór zawiera dane dotyczące charakterystki ziaren. Zawiera 3 klasy ziaren: Kama, Rosa and Canadian, po 70 rekordów każdy. Każde z ziaren opisane jest przez 7 atrybutów rzeczywistych. Łącznie 210 rekordów. Każde z ziaren jest opisane, nie posiada brakujących wartości. Zbiór może być wykorzystywany do metod klasyfikacji oraz klasteryzacji.
	\subsection{Dystrybucja klas}
Liczności poszczególnych klas zostały przedstawione w tabeli \ref{dist-seeds}
\begin{table}[H]
\centering
\caption{Dystrybucja klas w zbiorze seeds.}
\label{dist-seeds}
\begin{tabular}{ll}
NAZWA    & LICZNOŚĆ \\
Kama     & 70       \\
Rosa     & 70       \\
Canadian & 70      
\end{tabular}
\end{table}
	\subsection{Atrybuty}
	Poniżej \ref{attr-seeds} przedstawiono opis poszczególnych atrybutów.
\begin{table}[H]
\centering
\caption{Opis atrybutów w zbiorze seeds.}
\label{attr-seeds}
\begin{tabular}{lll}
NAZWA    	& RODZAJ & OPIS \\
Wielkość	& REAL	& area $A$\\
Perymetr	& REAL & perimeter $P$\\
Kompaktowość & REAL & compactness C = $4*pi*A/P^2$\\
Długość		& REAL & length of kernel\\
Szerokość	& REAL & width of kernel\\
Współczynnik asymetrii	& REAL & asymmetry coefficient\\
Długość rowka			& REAL & length of kernel groove 
\end{tabular}
\end{table}

\section{Ecoli}
	\subsection{Opis}
Zbiór zawiera dane dotyczące miejsca lokalizacji białek. Zawiera 8 klas lokalizacji przedstawione w tabeli \ref{dist-ecoli}. Każda z lokalizacji opisana jest przez 5 atrybutów rzeczywistych, 2 binarne, 1 kategoryczny, z czego ostatni jest nazwą konkretnej sekwencji (unikalna dla każdego rekordu). Łącznie 336 rekordów. Każdy z nich jest w pełni opisany, nie posiada brakujących wartości. Zbiór może być wykorzystywany do metod klasyfikacji.
	\subsection{Dystrybucja klas}
	Liczności poszczególnych klas zostały przedstawione w tabeli \ref{dist-ecoli}
\begin{table}[H]
\centering
\caption{Dystrybucja klas w zbiorze ecoli.}
\label{dist-ecoli}
\begin{tabular}{lll}
NAZWA & LICZNOŚĆ & OPIS \\
cp	& 143	& cytoplasm\\
im	& 77	& inner membrane without signal sequence\\
pp	& 52	& perisplasm\\
imU	& 35	& inner membrane, uncleavable signal sequence\\
om	& 20	& outer membrane\\
omL	& 5	 & outer membrane lipoprotein\\
imL	& 2	 & inner membrane lipoprotein\\
imS	& 2	 & inner membrane, cleavable signal sequence
\end{tabular}
\end{table}
	\subsection{Atrybuty}
	Poniżej \ref{attr-ecoli} przedstawiono opis poszczególnych atrybutów.
\begin{table}[H]
\centering
\caption{Opis atrybutów w zbiorze seeds.}
\label{attr-ecoli}
\begin{tabular}{lll}
NAZWA    	& RODZAJ \\
name	& UNIQUE	\\
mcg	& REAL \\
gvh & REAL \\
lip		& BINARY \\
chg	& BINARY \\
aac	& REAL \\
alm2	& REAL \\
alm1	& REAL
\end{tabular}
\end{table}
\newpage
\section{User Knwoledge Modelling}
	\subsection{Klasy}
Zbiór zawiera dane dotyczące koorelacji między wynikami testów badanych osób, a czasem spędzonym na naukę. Zawiera 4 klasy oznaczające wynik egzaminu, przedstawione w tabeli \ref{dist-ukm}. Każda z lokalizacji opisana jest przez 5 atrybutów rzeczywistych.
Łącznie 403 rekordów. Każdy z nich jest w pełni opisany, nie posiada brakujących wartości. Zbiór może być wykorzystywany do metod klasyfikacji oraz klasteryzacji.
	\subsection{Dystrybucja klas}
		Liczności poszczególnych klas zostały przedstawione w tabeli \ref{dist-ukm}
\begin{table}[H]
\centering
\caption{Dystrybucja klas w zbiorze seeds.}
\label{dist-ukm}
\begin{tabular}{ll}
NAZWA    & LICZNOŚĆ \\
very-low     & 50       \\
low     & 129       \\
middle & 122 \\
high & 130      
\end{tabular}
\end{table}
	\subsection{Atrybuty}
		Poniżej \ref{attr-ukm} przedstawiono opis poszczególnych atrybutów.
\begin{table}[H]
\centering
\caption{Opis atrybutów w zbiorze ukm.}
\label{attr-ukm}
\begin{tabular}{lll}
NAZWA    	& RODZAJ & OPIS \\
STG & REAL & Stopień poświęcenia czasu uczenia na główny cel\\
SCG & REAL & Stopień powtarzania informacji o głównym celu\\
STR & REAL & Stopień poświęcenia czasu uczenia na elementy powiązane z głównym celem\\
LPR & REAL & Wynik egzaminu powiązanego z głównym celem \\
PEG & REAL & Wynik egzaminu z głównym celem
\end{tabular}
\end{table}	
