\chapter{Podsumowanie}
Zagadnienie wyszukiwania obrazów i filmów w sieci Internet jest bardzo złożone pod wieloma względami. Pierwszym z nich może być złożoność obliczeniowa jaka wiąże się z przetwarzaniem obrazów. Kolejnym typem złożoności jest złożoność pamięciowa, która dotyczy indeksacji i skatalogowania wielu plików audiowizualnych w celu ich wyszukiwania. 

Wyszukiwanie obrazów podzielić można na dwie grupy - wyszukiwanie treścią i wyszukiwanie tekstem. Pierwsze z nich opiera się na przetwarzaniu obrazów i wymaga dużej mocy obliczeniowej. Drugie zaś zbliżone jest do wyszukiwania tekstu  w sieci. W obu przypadkach stosowane są zaawansowane algorytmy, a od niedawna metody sztucznej inteligencji.

Aby wyszukiwanie obrazów było szybsze i efektywniejsze stosuje się standard opisujący metadane plików audiowizualnych - MPEG-7. Charakteryzuje on szereg deskryptorów, które pozwalają na dokładny opis plików multimedialnych. Dzięki niemu przechowywanie informacji o multimediach zostało ujednolicone i usystematyzowane, przez co tworzenie aplikacji do wyszukiwania treści stało się prostsze.

W przyszłości zastosowanie sztucznej inteligencji i metod uczenia maszynowego znacznie polepszy wyszukiwarki obrazów i filmów. Przykładem może tu być wcześniej opisany autoenkoder, który znacznie upraszcza analizę obrazu i przenosi ciężar  skomplikowanego przetwarzania obrazu na sztuczne sieci neuronowe i proces uczenia.