\chapter{Wnioski}
\section{Dyskretyzacja}
W przypadku zbioru seeds, dyskretyzacja obiema metodami nie przyniosła znacznych różnic w skuteczności klasyfikatorów. Obie metody wykazały się podobną skutecznością. Największą dokładność (accuracy) uzyskano dla dyskretyzacji wszystkich atrybutów pomiędzy 14 wartości. Wyniosła ona odpowiednio 93.7\% dla metody częstotliwościowej oraz 92.7\% dla metody interwałowej. W przypadku zbyt niskiej ilości przedziałów dyskretyzacji (np. 2,3) odnotowano znaczący spadek skuteczności klasyfikatorów, co jest zgodne z intuicją - klasyfikacja będzie skuteczniejsza gdy wartości atrybutów będą niosły ze sobą większą ilość informacji.

W zbiorze ecoli dla precyzji (precision) oraz fscore wielokrotnie wystąpiła wartość 'nan' oznaczająca dosłownie nie-liczbę. W przypadku precyzji wartość ta oznacza, że dla danej klasy nie 
\section{Kroswalidacja}
\section{Ocena klasyfikatorów}
\section{Rozkład normalny}